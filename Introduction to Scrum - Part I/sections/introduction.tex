\section{Introduction}
\subsection{}

\begin{frame}
	\frametitle{Agile Software Development}
	\begin{itemize}
		\setlength\itemsep{0.7em}
		\item In the late 1990’s, several new software development methodologies emerged
		\item Emphasized close collaboration between development team and business stakeholders
		\item Focused on frequent delivery of business value
	\end{itemize}
\end{frame}

\begin{frame}
	\frametitle{Agile Software Development}
	\begin{itemize}
		\item The term ``Agile'' appeared for the first time in 2001 when the \textit{Agile Manifesto} was published
	\end{itemize}
\end{frame}

\begin{frame}
	\frametitle{Agile Manifesto}
	\begin{itemize}
		\setlength\itemsep{0.7em}
		\item \textbf{Individuals and interactions} over processes and tools
		\item \textbf{Working software} over comprehensive documentation
		\item \textbf{Customer collaboration} over contract negotiation
		\item \textbf{Responding to change} over following a plan
	\end{itemize}
\end{frame}

\begin{frame}
	\frametitle{Scrum Framework}
	\visible<1->{
		\begin{block}{Definition}
			\textit{Scrum (n): A framework within which people can address complex adaptive problems, while productively and creatively delivering products of the highest possible value.}
		\end{block}
	}
	\visible<2->{
		\vspace{1em}
		Key Characteristics:
		\begin{itemize}
			\item Lightweight
			\item Simple to understand
			\item Difficult to master
		\end{itemize}
	}
\end{frame}

\begin{frame}
	\frametitle{Scrum Framework}
	\visible<1->{
		Main concept:
		\begin{itemize}
			\item Have a small team of people that is highly \textit{flexible} and \textit{adaptive}
			\item Employ an iterative, incremental approach to \textit{optimize predictability} and \textit{control risk}
			\item Make decisions based on empirical process control theory, or \textit{empiricism}
		\end{itemize}
	}
	\visible<2->{
		\vspace{1em}
		\begin{block}{Definition}
			\textit{Empiricism (n): Knowledge comes from experience and making decisions on what is known.}
		\end{block}
	}
\end{frame}

\begin{frame}
	\frametitle{Scrum Framework}
	Three pillars uphold every implementation of empirical process control:
	\vspace{1em}
	\begin{itemize}
		\setlength\itemsep{0.7em}
		\item<1-> \textbf{Transparency}\\
		Make significant aspects of the process visible to those responsible for the outcome
		\item<2-> \textbf{Inspection}\\
		Frequently inspect the progress towards a goal to detect undesirable variances
		\item<3-> \textbf{Adaption}\\
		Adjust the process as soon as possible to minimize further deviation
	\end{itemize}
\end{frame}