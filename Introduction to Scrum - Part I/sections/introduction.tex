\section{Introduction}
\subsection{}

\begin{frame}
	\frametitle{Agile Software Development}
	\begin{itemize}
		\setlength\itemsep{0.7em}
		\item In the late 1990’s, several new software development methodologies emerged
		\item Emphasized close collaboration between development team and business stakeholders
		\item Focused on frequent delivery of business value
	\end{itemize}
\end{frame}

\begin{frame}
	\frametitle{Agile Software Development}
	\begin{itemize}
		\item The term ``Agile'' appeared for the first time in 2001 when the \textit{Agile Manifesto} was published
	\end{itemize}
\end{frame}

\begin{frame}
	\frametitle{Agile Manifesto}
	\begin{itemize}
		\setlength\itemsep{0.7em}
		\item \textbf{Individuals and interactions} over processes and tools
		\item \textbf{Working software} over comprehensive documentation
		\item \textbf{Customer collaboration} over contract negotiation
		\item \textbf{Responding to change} over following a plan
	\end{itemize}
\end{frame}

\begin{frame}
	\frametitle{Scrum Framework}
	\visible<1->{
		\begin{block}{Definition}
			\textit{Scrum: A lightweight framework that helps people, teams and organizations generate value through adaptive solutions for complex problems.}
		\end{block}
	}
	\visible<2->{
		\vspace{0.5em}
		Key Characteristics:
		\begin{itemize}
			\item Simple
			\item Purposefully incomplete
			\item Built upon by the collective intelligence of the people using it
			\item Follows an iterative, incremental approach to \textit{optimize predictability} \& \textit{control risk}
			\item Makes visible the relative efficacy of current management, environment and work techniques, so that \textit{improvements} can be made
		\end{itemize}
	}
\end{frame}

\begin{frame}
	\frametitle{Scrum Framework}
		\textbf{Main concept}\\
		\vspace{0.5em}
		A Scrum Master creates an environment where:
		\vspace{0.5em}
		\begin{itemize}
			\item A Product Owner orders the work for a complex problem into a Product Backlog
			\item The Scrum Team turns a selection of the work into an Increment of value during a Sprint
			\item The Scrum Team and its stakeholders inspect the results and adjust for the next Sprint
			\item \textit{Repeat}
		\end{itemize}
\end{frame}

\begin{frame}
	\frametitle{Scrum Framework}
	Scrum is founded on:
	\vspace{1em}
	\begin{itemize}
		\setlength\itemsep{0.7em}
		\item<1-> \textbf{Empiricism}\\
		Knowledge comes from experience and making decisions based on what is observed
		\item<2-> \textbf{Lean thinking}\\
		Reduce waste (non-value added activities) and focus on the essentials
	\end{itemize}
\end{frame}

\begin{frame}
	\frametitle{Scrum Framework}
	Three pillars uphold every implementation of Scrum:
	\vspace{1em}
	\begin{itemize}
		\setlength\itemsep{0.7em}
		\item<1-> \textbf{Transparency}\\
		Make the emergent process and work visible to those performing the work as well as those receiving the work
		\item<2-> \textbf{Inspection}\\
		Frequently inspect the Scrum artifacts and progress toward agreed goals to detect potentially undesirable variances or problems
		\item<3-> \textbf{Adaption}\\
		If any aspects of a process deviate outside acceptable limits or if the resulting product is unacceptable adjustments must be made \textit{as soon as possible} to minimize further deviation
	\end{itemize}
\end{frame}