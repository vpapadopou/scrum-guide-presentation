\section{Roles}
\subsection{}

\begin{frame}
	\frametitle{The Scrum Team}
	
	The Scrum Team consists of:
    \vspace{1em}
    \begin{itemize}
    	\setlength\itemsep{0.7em}
    	\item A Product Owner
    	\item The Development Team
    	\item A Scrum Master
    \end{itemize}
\end{frame}

\begin{frame}
	\frametitle{The Scrum Team}

	\begin{columns}
		\begin{column}{0.5\textwidth}
			Scrum Team Characteristics:
			\vspace{1em}
			\begin{itemize}
				\setlength\itemsep{0.7em}
				\item Deliver products \textit{iteratively} and \textit{incrementally} to maximize opportunities for feedback
				\item \textit{Self-organizing}
				\item \textit{Cross-functional}
			\end{itemize}
		\end{column}
		\begin{column}{0.5\textwidth}
			\visible<2->{
				\begin{block}{Definition}
					\textit{Self-organizing teams choose how best to accomplish their work, rather than being directed by others outside the team.}
				\end{block}
				\begin{block}{Definition}
					\textit{Cross-functional teams have all competencies needed to accomplish the work without depending on others not part of the team.}
				\end{block}
			}
		\end{column}
	\end{columns}	
\end{frame}

\begin{frame}
	\frametitle{The Product Owner}
	
	\begin{columns}
		\begin{column}{0.7\textwidth}
			\begin{itemize}
        		\setlength\itemsep{0.7em}
        		\item Responsible for maximizing the value of the product resulting from work of the development team
        		\item The \textbf{only} person responsible for managing the \textit{Product Backlog}
        		\item \textbf{One person}, not a committee
        	\end{itemize}
		\end{column}
		\begin{column}{0.3\textwidth}
		    \vspace{-2em}
			\begin{figure}
                \includegraphics[width=1.6in]{images/po.jpg}
            \end{figure}
		\end{column}
	\end{columns}
\end{frame}

\begin{frame}
	\frametitle{The Product Owner}
	
	\begin{itemize}
		\setlength\itemsep{0.7em}
		\item The Product Owner's decisions are visible in the content and ordering of the Product Backlog
		\item They may have someone else manage the Product Backlog however, they remain \textbf{accountable}
	\end{itemize}
\end{frame}

\begin{frame}
	\frametitle{The Development Team}
	
	\begin{columns}
		\begin{column}{0.6\textwidth}
			\begin{itemize}
        		\setlength\itemsep{0.7em}
        		\item Consists of professionals who do the work of delivering a \textit{potentially releasable} Increment of "Done" product at the end of each Sprint 
        		\item Size: 3 to 9 members
        	\end{itemize}
		\end{column}
		\begin{column}{0.4\textwidth}
		    \vspace{-2em}
			\begin{figure}
                \includegraphics[width=2.4in]{images/devteam.jpg}
            \end{figure}
		\end{column}
	\end{columns}
\end{frame}

\begin{frame}
	\frametitle{The Development Team}
	Key Characteristics:
	\vspace{1em}
	\begin{itemize}
		\setlength\itemsep{0.7em}
		\item<1-> Self-Organizing \& Cross-Functional
		\item<1-> Scrum recognizes no titles for Development Team members
		\item<1-> Scrum recognizes no sub-teams in the Development Team
	\end{itemize}
	\visible<2->{
		\vspace{2em}
		\begin{alertblock}{Important}
			Accountability belongs to the team as a whole
		\end{alertblock}
	}
\end{frame}

\begin{frame}
	\frametitle{The Scrum Master}
	
	\begin{columns}
		\begin{column}{0.7\textwidth}
			\begin{itemize}
        		\setlength\itemsep{0.7em}
        		\item Promotes and supports Scrum
        		\item Helps everyone understand Scrum theory, practices and rules 
        		\item Helps those outside the Scrum Team understand which interactions are helpful and which aren't
        		\item Acts as a \textit{Servant-Leader} for the Scrum Team
        	\end{itemize}
		\end{column}
		\begin{column}{0.3\textwidth}
		    \vspace{-2em}
			\begin{figure}
                \includegraphics[width=1.6in]{images/sm.jpg}\\
            \end{figure}
		\end{column}
	\end{columns}
\end{frame}