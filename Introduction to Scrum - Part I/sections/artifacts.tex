\section{Artifacts}
\subsection{}

\begin{frame}
	\frametitle{Scrum Artifacts}
	Scrum uses the following Artifacts to provide transparency and opportunities for inspection and adaption:
	\vspace{1em}
	\begin{enumerate}
		\setlength\itemsep{0.7em}
		\item Product Backlog \visible<3->{- \textit{Commitment: Product Goal}}
		\item Sprint Backlog \visible<3->{- \textit{Commitment: Sprint Goal}}
		\item Increment \visible<3->{- \textit{Commitment: Definition of Done}}
	\end{enumerate}
	\visible<2->{
		\vspace{1em}
		\begin{alertblock}{Important}
			Each artifact contains a commitment to ensure it provides information that enhances transparency and focus against which progress can be \textbf{measured}
		\end{alertblock}
	}
\end{frame}

\begin{frame}
	\frametitle{Product Backlog}
	\begin{itemize}
		\setlength\itemsep{0.7em}
		\item Emergent, ordered list of what is needed to improve the product
		\item Single source of work undertaken by the Scrum Team
		\item Product Backlog items that can be Done by the Scrum Team within one Sprint are deemed \textit{ready} for selection in a Sprint Planning event
		\item Product Backlog items \textit{usually} acquire the necessary degree of transparency after refining activities
	\end{itemize}
\end{frame}

\begin{frame}
	\frametitle{Product Backlog Refinement}
	\begin{block}{Key Concept}
		The act of breaking down and further defining Product Backlog items into smaller, more precise items
	\end{block}
	\vspace{0.5em}
	\begin{itemize}
		\setlength\itemsep{0.7em}
		\item An ongoing activity during which details such as a \textit{description}, \textit{order} and \textit{size} are added to Product 	Backlog items 
		\item Attributes added usually vary with the domain of work
	\end{itemize}
\end{frame}

\begin{frame}
	\frametitle{Product Goal}
	\begin{itemize}
		\setlength\itemsep{0.7em}
		\item Future state of the product 
		\item Serves as a long-term objective for the Scrum Team to plan against
		\item Resides in the Product Backlog. The rest of the Product Backlog emerges to define ``what'' will fulfill the Product Goal
		\item The Scrum Team must fulfill (or abandon) one Product Goal before taking on the next
	\end{itemize}
\end{frame}

\begin{frame}
	\frametitle{Sprint Backlog}
	\begin{itemize}
		\setlength\itemsep{0.7em}
		\item The Sprint Goal (why)
		\item The set of Product Backlog items selected for the Sprint (what)
		\item An actionable plan for delivering the Increment (how)
	\end{itemize}
	\visible<2->{
		\vspace{1em}
		\begin{alertblock}{Important}
			A plan \textbf{by and for} the Developers that is updated throughout the Sprint as more is learned
		\end{alertblock}
	}
\end{frame}

\begin{frame}
	\frametitle{Sprint Goal}
	\begin{itemize}
		\setlength\itemsep{0.7em}
		\item Single objective for the Sprint
		\item Creates coherence and focus
		\item Encourages the Scrum Team to work together rather than on separate initiatives
		\item Provides flexibility to the Developers in terms of the exact work needed to achieve it
	\end{itemize}
\end{frame}

\begin{frame}
	\frametitle{Increment}
	\begin{itemize}
		\setlength\itemsep{0.7em}
		\item A concrete step toward the Product Goal
		\item The moment a Product Backlog item meets the Definition of Done, an Increment is born
		\item Each Increment is additive to all prior Increments and thoroughly verified to ensure that all Increments work together
		\item Multiple Increments may be created within a Sprint
		\item Must be \textbf{usable}
	\end{itemize}
\end{frame}

\begin{frame}
	\frametitle{Definition of Done}
	\begin{itemize}
		\setlength\itemsep{0.7em}
		\item A formal description of the state of the Increment when it meets the quality measures required for the product
		\item Creates transparency by providing everyone a shared understanding of what work was completed as part of the Increment
		\item Developers are required to conform to the Definition of Done
		\item Work cannot be considered part of an Increment unless it meets the Definition of Done
	\end{itemize}
\end{frame}