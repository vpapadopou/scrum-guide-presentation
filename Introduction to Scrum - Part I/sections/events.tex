\section{Events}
\subsection{}

\begin{frame}
	\frametitle{Scrum Events}
	\begin{columns}
		\begin{column}{0.5\textwidth}
			Scrum prescribes five formal events:
			\vspace{1em}
			\begin{enumerate}
				\setlength\itemsep{0.7em}
				\item The Sprint
				\item Sprint Planning
				\item Daily Scrum
				\item Sprint Review
				\item Sprint Retrospective
			\end{enumerate}
		\end{column}
		\begin{column}{0.5\textwidth}
			\visible<2->{
				\begin{block}{Key Characteristics}
					\begin{itemize}
						\setlength\itemsep{0.7em}
						\item Enable the transparency required
						\item Create regularity
						\item Minimize the need for meetings not defined in Scrum
						\item Each event is a formal opportunity to inspect and adapt Scrum artifacts
					\end{itemize}
				\end{block}
			}
		\end{column}
	\end{columns}	
\end{frame}

\begin{frame}
	\frametitle{The Sprint}
	\begin{itemize}
		\setlength\itemsep{0.7em}
		\item Acts as a container for all other events
		\item Duration: One month or less (consistency is important)
		\item A new sprint starts immediately after the conclusion of the previous Sprint
		\item Sprints enable predictability by ensuring inspection and adaptation of progress toward a Product Goal \textit{at least} every calendar month
	\end{itemize}
\end{frame}

\begin{frame}
	\frametitle{The Sprint}
	During The Sprint:
	\vspace{1em}
	\begin{itemize}
		\setlength\itemsep{0.7em}
		\item No changes are made that would endanger the Sprint Goal
		\item Quality does not decrease
		\item The Product Backlog is refined as needed
		\item Scope may be clarified and renegotiated with the Product Owner as more is learned
	\end{itemize}
\end{frame}

\begin{frame}
	\frametitle{The Sprint}
	Cancelling a Sprint:
	\begin{itemize}
		\setlength\itemsep{0.7em}
		\item \textbf{Only} the Product Owner has the authority to cancel a Sprint
		\item The Sprint is cancelled if the Sprint Goal becomes obsolete
	\end{itemize}
\end{frame}

\begin{frame}
	\frametitle{Sprint Planning}
	\begin{block}{Key Concept}
		Sprint Planning initiates the Sprint by laying out the work to be performed
	\end{block}
	\vspace{0.5em}
	\begin{itemize}
		\setlength\itemsep{0.7em}
		\item Why is this Sprint valuable?
	  \item What can be Done this Sprint?
	  \item How will the chosen work get done?
		\item Max duration: 8 hours for a one-month Sprint
		\item Attendees: All Scrum Team members
	\end{itemize}
\end{frame}

\begin{frame}
	\frametitle{Sprint Planning}
	\begin{itemize}
		\setlength\itemsep{0.7em}
		\item The Product Owner ensures that attendees are \textit{prepared} to discuss the most important Product Backlog items and how they map to the Product Goal
		\item The Scrum Team may refine Product Backlog items during Sprint Planning to increase its understanding and confidence
		\item Developers will become more confident in their Sprint forecasts as they learn more about their performance, upcoming capacity and the Definition of Done
	\end{itemize}
\end{frame}

\begin{frame}
	\frametitle{Sprint Planning}
	\begin{columns}
		\begin{column}{0.5\textwidth}
			\begin{itemize}
				\setlength\itemsep{0.7em}
				\item The Scrum Team may invite other people to attend the Sprint Planning to provide advice
				\item It is solely up to the Developers to plan how to turn Product Backlog items to an Increment that meets the Definition of Done
				\item Output of Sprint Planning: \textit{Sprint Backlog}
			\end{itemize}
		\end{column}
		\begin{column}{0.5\textwidth}
			\visible<2->{
				\begin{block}{Sprint Backlog}
					\begin{itemize}
						\item The Sprint Goal\\
						(communicates why the Sprint is valuable to stakeholders)
						\item The Product Backlog items selected for the Sprint
						\item A plan for delivering them
					\end{itemize}
				\end{block}
			}
		\end{column}
	\end{columns}	
\end{frame}

\begin{frame}
	\frametitle{Daily Scrum}
	\begin{block}{Key Concept}
		Inspect progress toward the Sprint Goal and adapt the Sprint Backlog as necessary
	\end{block}
	\vspace{0.5em}
	\begin{itemize}
		\setlength\itemsep{0.7em}
		\item Held every working day of the Sprint at the same place and time
		\item Max duration: 15 minutes
		\item Attendees: All Developers
	\end{itemize}
\end{frame}

\begin{frame}
	\frametitle{Daily Scrum}
	\begin{itemize}
		\setlength\itemsep{0.7em}
		\item If the Product Owner or Scrum Master are actively working on items in the Sprint Backlog, they participate as Developers
		\item Developers can select whatever structure and techniques they want as long as their Daily Scrum focuses on progress toward the Sprint Goal and produces an actionable plan for the next day of work
	\end{itemize}
	\visible<2->{
		\vspace{0.5em}
		\begin{alertblock}{Important}
			The Daily Scrum is not the only time Developers are allowed to adjust their plan. They often meet throughout the day for more detailed discussions about adapting or re-planning the rest of the Sprint's work
		\end{alertblock}
	}
\end{frame}

\begin{frame}
	\frametitle{Daily Scrum}
	\begin{exampleblock}{Benefits}
		\begin{itemize}
			\item Improve communication
			\item Identify impediments
			\item Promote quick decision-making
			\item Eliminate the need for other meetings
		\end{itemize}
	\end{exampleblock}
\end{frame}

\begin{frame}
	\frametitle{Sprint Review}
	\begin{block}{Key Concept}
		Inspect the outcome of the Sprint and determine future adaptations
	\end{block}
	\vspace{0.5em}
	\begin{itemize}
		\setlength\itemsep{0.7em}
		\item Held at the end of the Sprint, before the Sprint Retrospective
		\item Max duration: 4 hours for a one-month Sprint
		\item Attendees: All Scrum Team members and Stakeholders
	\end{itemize}
\end{frame}

\begin{frame}
	\frametitle{Sprint Review}
	\begin{itemize}
		\setlength\itemsep{0.7em}
		\item The Sprint Review is a \textit{working session} and the Scrum Team should avoid limiting it to a presentation
		\item The Product Backlog may be adjusted following the Sprint Review to meet new opportunities
	\end{itemize}
\end{frame}

\begin{frame}
	\frametitle{Sprint Retrospective}
	\begin{block}{Key Concept}
		\begin{itemize}
			\item Inspect how the last Sprint went with regards to individuals, interactions, processes, tools and the Definition of Done
			\item Identify the most helpful changes and \textit{address} them as soon as possible
		\end{itemize}
	\end{block}
	\vspace{0.5em}
	\begin{itemize}
		\setlength\itemsep{0.7em}
		\item Last event of the Sprint
		\item Max duration: 3 hours for a one-month Sprint
		\item Attendees: All Scrum Team members
	\end{itemize}
\end{frame}