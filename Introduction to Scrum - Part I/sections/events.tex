\section{Events}
\subsection{}

\begin{frame}
	\frametitle{Scrum Events}
	\begin{columns}
		\begin{column}{0.5\textwidth}
			Scrum prescribes five formal events:
			\vspace{1em}
			\begin{enumerate}
				\setlength\itemsep{0.7em}
				\item The Sprint
				\item Sprint Planning
				\item Daily Scrum
				\item Sprint Review
				\item Sprint Retrospective
			\end{enumerate}
		\end{column}
		\begin{column}{0.5\textwidth}
			\visible<2->{
				\begin{block}{Key Characteristics}
					\begin{itemize}
						\setlength\itemsep{0.7em}
						\item Create regularity
						\item Minimize the need for meetings not defined in Scrum
						\item Time-boxed
						\item Each event is an opportunity to inspect and adapt something
					\end{itemize}
				\end{block}
			}
		\end{column}
	\end{columns}	
\end{frame}

\begin{frame}
	\frametitle{The Sprint}
	\begin{itemize}
		\setlength\itemsep{0.7em}
		\item Acts as a container for all other events
		\item Duration: One month or less (consistency is important)
		\item A new sprint starts immediately after the conclusion of the previous Sprint
		\item A ``Done'', potentially releasable product increment is created
	\end{itemize}
\end{frame}

\begin{frame}
	\frametitle{The Sprint}
	During The Sprint:
	\vspace{1em}
	\begin{itemize}
		\setlength\itemsep{0.7em}
		\item No changes that would endanger the Sprint Goal
		\item Quality goals do not decrease
		\item Scope may be clarified and re-negotiated between the Product Owner and the Development Team as more is learned
	\end{itemize}
\end{frame}

\begin{frame}
	\frametitle{The Sprint}
	Cancelling a Sprint:
	\begin{itemize}
		\setlength\itemsep{0.7em}
		\item \textbf{Only} the Product Owner has the authority to cancel a Sprint
		\item The Sprint is cancelled if the Sprint Goal becomes obsolete
	\end{itemize}
\end{frame}

\begin{frame}
	\frametitle{Sprint Planning}
	\begin{block}{Key Concept}
		During Sprint Planning, the Scrum Team collaborates and creates a plan for the next Sprint
	\end{block}
	\vspace{0.5em}
	\begin{itemize}
		\setlength\itemsep{0.7em}
		\item What can be delivered in the Increment resulting from the upcoming Sprint?
		\item How will the work needed to deliver the Increment be achieved?
		\item Max duration: 8 hours for a one-month Sprint
		\item Attendees: All Scrum Team members
	\end{itemize}
\end{frame}

\begin{frame}
	\frametitle{Sprint Planning}
	\begin{itemize}
		\setlength\itemsep{0.7em}
		\item The number of items selected from the Product Backlog for the Sprint is \textbf{solely} up to the Development Team
		\item The Product Owner can help to clarify selected Product Backlog Items and make trade-offs
		\item The Development Team may renegotiate selected Product Backlog Items with the Product Owner
	\end{itemize}
\end{frame}

\begin{frame}
	\frametitle{Sprint Planning}
	\begin{columns}
		\begin{column}{0.5\textwidth}
			\begin{itemize}
				\setlength\itemsep{0.7em}
				\item The Development Team may invite other people to attend to provide technical or domain advice
				\item Output of Sprint Planning: \textit{Sprint Backlog} and \textit{Sprint Goal}
			\end{itemize}
		\end{column}
		\begin{column}{0.5\textwidth}
			\visible<2->{
				\begin{block}{Sprint Backlog}
					A set of Product Backlog Items selected for this Sprint plus, a plan for delivering them
				\end{block}
				\begin{block}{Sprint Goal}
					An objective that will be met within the Sprint through the implementation of the selected Product Backlog Items
				\end{block}
			}
		\end{column}
	\end{columns}	
\end{frame}

\begin{frame}
	\frametitle{Daily Scrum}
	\begin{block}{Key Concept}
		During the Daily Scrum, the Development Team plans work for the next 24 hours
	\end{block}
	\vspace{0.5em}
	\begin{itemize}
		\setlength\itemsep{0.7em}
		\item Held every day of the Sprint at the same place and time
		\item Max duration: 15 minutes
		\item Attendees: All Development Team members
	\end{itemize}
\end{frame}

\begin{frame}
	\frametitle{Daily Scrum}
	\begin{itemize}
		\setlength\itemsep{0.7em}
		\item The Daily Scrum is an internal meeting for the Development Team. If others are present, the Scrum Master ensures they \textbf{do not disrupt} the meeting
		\item The Development Team or team members often meet immediately after the Daily Scrum for related discussions
	\end{itemize}
	\visible<2->{
		\vspace{1em}
		\begin{alertblock}{Important}
			The Scrum Master ensures that the Development Team has the meeting but the Development Team is responsible for conducting the Daily Scrum
		\end{alertblock}
	}
\end{frame}

\begin{frame}
	\frametitle{Daily Scrum}
	\begin{exampleblock}{Benefits}
		\begin{itemize}
			\item Improve communications
			\item Eliminate other meetings
			\item Identify impediments to development for removal
			\item Highlight and promote quick decision-making
			\item Improve the Development Team's level of knowledge
		\end{itemize}
	\end{exampleblock}
\end{frame}

\begin{frame}
	\frametitle{Sprint Review}
	\begin{block}{Key Concept}
		\begin{itemize}
			\item Inspect the Increment and adapt the Product Backlog if needed
			\item Collaborate on the next things that could be done to optimize value
		\end{itemize}
	\end{block}
	\vspace{0.5em}
	\begin{itemize}
		\setlength\itemsep{0.7em}
		\item Held at the end of each Sprint
		\item Max duration: 4 hours for a one-month Sprint
		\item Attendees: All Scrum Team members and Stakeholders\\
		(Invited by the Product Owner)
	\end{itemize}
\end{frame}

\begin{frame}
	\frametitle{Sprint Review}
	\begin{itemize}
		\setlength\itemsep{0.7em}
		\item The Sprint Review is \textbf{not} a demo
		\item The presentation of the Increment is intended to elicit feedback and foster collaboration
		\item Result: A revised Product Backlog that defines the \textit{probable} Product Backlog Items for the next Sprint
	\end{itemize}
	\visible<2->{
		\vspace{1em}
		\begin{alertblock}{Important}
			The Sprint Review is an informal meeting, not a status meeting
		\end{alertblock}
	}
\end{frame}

\begin{frame}
	\frametitle{Sprint Retrospective}
	\begin{block}{Key Concept}
		\begin{itemize}
			\item Identify how the last Sprint went with regards to people, relationships, processes and tools
			\item Identify improvements and create a plan for implementing them
		\end{itemize}
	\end{block}
	\vspace{0.5em}
	\begin{itemize}
		\setlength\itemsep{0.7em}
		\item Occurs after the Sprint Review and prior to the next Sprint Planning
		\item Max duration: 3 hours for a one-month Sprint
		\item Attendees: All Scrum Team members
	\end{itemize}
\end{frame}

\begin{frame}
	\frametitle{Sprint Retrospective}
	\begin{itemize}
		\setlength\itemsep{0.7em}
		\item The Scrum Master participates as peer team member from the accountability over the Scrum process
		\item By the end of the Retrospective the Scrum Team should have identified improvements that it will implement in the next Sprint
	\end{itemize}
	\visible<2->{
		\vspace{1em}
		\begin{alertblock}{Important}
			The Scrum Master ensures that the meeting is positive and productive
		\end{alertblock}
	}
\end{frame}