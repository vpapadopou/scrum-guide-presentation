\section{Definition of Done}
\subsection{}

\begin{frame}
    \frametitle{Delivering ``Done'' Increments}
    \begin{itemize}
        \setlength\itemsep{0.7em}
        \item Development Teams deliver an Increment of product functionality every Sprint
        \item The Increment \textbf{must} be useable, should the Product Owner chooses to immediately release it
    \end{itemize}
\end{frame}

\begin{frame}
    \frametitle{Delivering ``Done'' Increments}
    \begin{columns}
        \begin{column}{0.5\textwidth}
            \begin{itemize}
                \setlength\itemsep{0.7em}
                \item When a Product Backlog Item or an Increment is described as ``Done'', everyone must understand what ``Done'' means
                \item Scrum Team members must have a shared understanding of what it means for work to be complete, to ensure transparency
            \end{itemize}
        \end{column}
        \begin{column}{0.5\textwidth}
            \vspace{-1em}
            \begin{figure}
                \includegraphics[width=2.7in]{images/itsdone.jpg}
            \end{figure}
        \end{column}
    \end{columns}
\end{frame}

\begin{frame}
    \frametitle{Definition of Done}
    \begin{itemize}
        \setlength\itemsep{0.7em}
        \item The definition of “Done” is used to assess when work is complete on the product Increment
        \item It also guides the Development Team in knowing how many Product Backlog Items it can select during a Sprint Planning
    \end{itemize}
    \visible<2-> {
        \vspace{1em}
        \begin{alertblock}{Important}
            If the definition of ``Done'' for an increment is part of the conventions, standards or guidelines of the development organization, all Scrum Teams must follow it as a minimum
        \end{alertblock}
    }
\end{frame}

\begin{frame}
    \frametitle{Definition of Done}
    \begin{itemize}
        \setlength\itemsep{0.7em}
        \item As Scrum Teams mature, it is expected that their definitions of ``Done'' will expand to include more stringent criteria for higher quality
        \item If there are multiple Scrum Teams, they must use the same definition of ``Done''
    \end{itemize}
\end{frame}