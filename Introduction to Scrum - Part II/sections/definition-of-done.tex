\section{Definition of Done}
\subsection{}

\begin{frame}
	\frametitle{Delivering Done Increments}
	\begin{itemize}
		\setlength\itemsep{0.7em}
		\item A formal description of the state of the Increment when it meets the quality measures required for the product
		\item Creates transparency by providing everyone a shared understanding of what work was completed as part of the Increment
		      \item<2-> Work cannot be considered part of an Increment unless it meets the Definition of Done
	\end{itemize}
\end{frame}

\begin{frame}
	\frametitle{Delivering Done Increments}
	\begin{columns}
		\begin{column}{0.5\textwidth}
			\begin{alertblock}{Important}
				If a Product Backlog item does not meet the Definition of Done by the end of the Sprint, it \textbf{cannot} be released or even presented at the Sprint Review. Instead, it returns to the Product Backlog for future consideration
			\end{alertblock}
		\end{column}
		\begin{column}{0.5\textwidth}
			\vspace{-1em}
			\begin{figure}
				\includegraphics[width=2.7in]{images/itsdone.jpg}
			\end{figure}
		\end{column}
	\end{columns}
\end{frame}

\begin{frame}
	\frametitle{Definition of Done - Notes}
	\begin{itemize}
		\setlength\itemsep{0.7em}
		\item If the Definition of Done is part of the standards of the organization all Scrum Teams must follow it as a \textit{minimum}
		\item If there is not an organizational standard the \textit{Scrum Team} must create a Definition of Done that is appropriate for the product
	\end{itemize}
	\visible<2-> {
		\vspace{1em}
		\begin{alertblock}{Important}
			Multiple Scrum Teams working together on a Product must \textit{mutually} define and comply with the same Definition of Done
		\end{alertblock}
	}
\end{frame}
