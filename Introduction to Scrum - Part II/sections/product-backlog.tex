\section{Product Backlog}
\subsection{}

\begin{frame}
    \frametitle{Product Backlog}
    \begin{itemize}
        \setlength\itemsep{0.7em}
        \item An ordered list of everything that is known to be needed in the product
        \item Lists all features, functions, requirements, enhancements and fixes that constitute the changes to be made to the product in future releases
        \item Higher  ordered Product Backlog Items are usually clearer and more detailed than lower ordered ones
    \end{itemize}
    \visible<2->{
		\vspace{1em}
		\begin{alertblock}{Important}
			Multiple Scrum Teams that work together on the same product use the same Product Backlog
		\end{alertblock}
	}
\end{frame}

\begin{frame}
    \frametitle{Product Backlog Items}
    \begin{columns}
        \begin{column}{0.5\textwidth}
            Product Backlog Items have the following attributes:
            \vspace{1em}
            \begin{enumerate}
                \setlength\itemsep{0.7em}
                \item Description
                \item Order
                \item Estimate
                \item Value
            \end{enumerate}
        \end{column}
        \begin{column}{0.5\textwidth}
            \visible<2->{
                \begin{block}{Additional Attributes}
                    Product Backlog Items often include test descriptions that will prove its completeness when ``Done''
                \end{block}
            }
            \visible<3->{
                \begin{block}{Refinement}
                    Refinement is the act of adding \textbf{detail}, \textbf{estimates}, and \textbf{order} to items in the Product Backlog
                \end{block}
            }
        \end{column}
    \end{columns}
\end{frame}

\begin{frame}
    \frametitle{Product Backlog Items}
    \begin{itemize}
        \setlength\itemsep{0.7em}
        \item Product Backlog items that will occupy the Development Team for the upcoming Sprint are refined so that any one item can reasonably be ``Done'' within the Sprint
        \item Product Backlog items that can be ``Done'' by the Development Team within one Sprint are deemed ``Ready'' for selection in a Sprint Planning
    \end{itemize}
    \vspace{1em}
    \visible<2->{
        \begin{block}{Estimates}
            The Development Team is \textit{responsible} for all estimates
        \end{block}
        \begin{alertblock}{Important}
            The Product Owner may influence the Development Team by helping it understand and select trade-offs, but the people who will perform the work make the final estimate
        \end{alertblock}
    }
\end{frame}