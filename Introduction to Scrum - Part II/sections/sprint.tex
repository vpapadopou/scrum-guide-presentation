\section{In a Sprint}
\subsection{}

\begin{frame}
	\frametitle{Sprint Planning}
	Sprint Planning answers the following:
	\vspace{1em}
	\begin{itemize}
		\setlength\itemsep{0.7em}
		\item \textbf{Why} is this Sprint valuable?
		\item \textbf{What} can be Done this Sprint?
		\item \textbf{How} will the chosen work get done?
	\end{itemize}
\end{frame}

\begin{frame}
    \frametitle{Sprint Planning}
    \begin{itemize}
        \setlength\itemsep{0.7em}
        \item<1-> \textbf{Step 1}\\
        The Product Owner proposes how the product could increase its value and utility in the current Sprint. The whole Scrum Team then collaborates to define a \textit{Sprint Goal}
        \item<2-> \textbf{Step 2}\\
        Through discussion with the Product Owner, the Developers select items from the Product Backlog to include in the current Sprint. The Scrum Team may refine these items during this process, which increases understanding and confidence
    \end{itemize}
\end{frame}

\begin{frame}
    \frametitle{Sprint Planning}
    \begin{itemize}
        \setlength\itemsep{0.7em}
        \item \textbf{Step 3}\\
        For each selected Product Backlog item, the Developers plan the work necessary to create an Increment that meets the Definition of Done
    \end{itemize}
    \visible<2-> {
        \vspace{1em}
        \begin{alertblock}{Important}
            How Developers will plan to turn Product Backlog items into Increments of value is at their \textbf{sole} discretion
        \end{alertblock}
    }
\end{frame}

\begin{frame}
    \frametitle{Sprint Planning}
    \begin{block}{Sprint Backlog}
        \begin{itemize}
            \setlength\itemsep{0.7em}
            \item The Sprint Goal (why)
            \item The set of Product Backlog items selected for the Sprint (what)
            \item An actionable plan for delivering the Increment (how)
        \end{itemize}
    \end{block}
    \visible<2-> {
        \vspace{0.2em}
        \begin{itemize}
            \setlength\itemsep{0.7em}
            \item Highly visible, \textit{real-time} picture of the work that the Developers plan to accomplish during the Sprint in order to achieve the Sprint Goal
            \item Updated throughout the Sprint as more is learned
            \item Should have enough detail so progress can be inspected during the Daily Scrum
        \end{itemize}
    }
\end{frame}

\begin{frame}
	\frametitle{Sprint Backlog}
	\begin{block}{Sprint Goal}
		\begin{itemize}
			\setlength\itemsep{0.7em}
			\item Single objective for the Sprint
            \item Creates coherence and focus
            \item Encourages the Scrum Team to work together rather than on separate initiatives
            \item Provides flexibility to the Developers in terms of the exact work needed to achieve it
		\end{itemize}
    \end{block}
\end{frame}

\begin{frame}
	\frametitle{Sprint Backlog}
	\begin{alertblock}{Important}
		As Developers work during the Sprint, they keep the Sprint Goal in mind. If the work turns out to be different than they expected, they collaborate with the Product Owner to negotiate the scope of the Sprint Backlog within the Sprint \textbf{without} affecting the Sprint Goal
	\end{alertblock}
\end{frame}

\begin{frame}
    \frametitle{Sprint Review}
    \begin{itemize}
        \setlength\itemsep{0.7em}
        \item The Scrum Team presents the results of their work to key stakeholders and progress toward the Product Goal is discussed
        \item Attendees then examine what has changed in their environment and collaborate on what to do next
        \item<2-> The Sprint Review is a \textbf{working session} and the Scrum Team should avoid limiting it to a presentation
	    \item<2-> The Product Backlog may be adjusted following the Sprint Review to meet new opportunities
    \end{itemize}
\end{frame}
