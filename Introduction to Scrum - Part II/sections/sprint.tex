\section{In a Sprint}
\subsection{}

\begin{frame}
    \frametitle{Sprint Planning}
    \begin{columns}
        \begin{column}{0.5\textwidth}
            Sprint Planning answers the following:
            \vspace{1em}
            \begin{itemize}
                \setlength\itemsep{0.7em}
                \item What can be delivered in the Increment resulting from the upcoming Sprint?
                \item How will the work needed to deliver the Increment be achieved?
            \end{itemize}
        \end{column}
        \begin{column}{0.5\textwidth}
            \visible<2->{
                \begin{block}{Meeting Input}
                    \begin{itemize}
                        \setlength\itemsep{0.7em}
                        \item The Product Backlog
                        \item The latest Product Increment
                        \item Projected capacity of the Development Team
                        \item Past Performance of the Development Team
                    \end{itemize}
                \end{block}
            }
        \end{column}
    \end{columns}
\end{frame}

\begin{frame}
    \frametitle{Sprint Planning}
    \begin{itemize}
        \setlength\itemsep{0.7em}
        \item<1-> \textbf{Step 1}\\
        The Product Owner discusses the objective that the Sprint should achieve and the Product Backlog Items that, if completed in the Sprint, would achieve the Sprint Goal
        \item<2-> \textbf{Step 2}\\
        The Development Team selects a number of items to be built during the Sprint and adds them to the \textit{Sprint Backlog}
        \item<3-> \textbf{Step 3}\\
        During Sprint Planning the Development team also crafts a \textit{Sprint Goal}
    \end{itemize}
\end{frame}

\begin{frame}
    \frametitle{Sprint Planning}
    \begin{block}{Sprint Goal}
        \begin{itemize}
            \setlength\itemsep{0.7em}
            \item An \textbf{objective} that will be met within the Sprint through the implementation of the Product Backlog
            \item Provides guidance to the Development Team on \textit{why} it is building the Increment
        \end{itemize}
    \end{block}
    \vspace{1em}
    \begin{itemize}
        \setlength\itemsep{0.7em}
        \item<2-> \textbf{Step 4}\\
        The Development Team decides \textit{how} it will build this functionality into a ``Done'' product Increment during the Sprint
    \end{itemize}
\end{frame}
    
\begin{frame}
    \frametitle{Sprint Planning}
    \begin{alertblock}{Important}
        The number of items selected from the Product Backlog for the Sprint is \textbf{solely} up to the Development Team
    \end{alertblock}
    \vspace{1em}
    \begin{itemize}
        \setlength\itemsep{0.7em}
        \item<2-> The Product Owner can help to clarify the selected Product Backlog items and make trade-offs
        \item<2-> If the Development Team determines it has too much or too little work, it may renegotiate the selected Product Backlog items with the Product Owner
    \end{itemize}
\end{frame}

\begin{frame}
    \frametitle{Cancelling a Sprint}
    \begin{itemize}
        \setlength\itemsep{0.7em}
        \item A Sprint should be cancelled if it no longer makes sense given the circumstances
        \item \textbf{Only} the Product Owner has the authority to cancel the Sprint
        \item They may however do so under influence from the stakeholders, the Development Team or the Scrum Master
    \end{itemize}
    \visible<2-> {
        \vspace{1em}
        \begin{alertblock}{Important}
            Sprint cancellations are often traumatic to the Scrum Team and should be avoided
        \end{alertblock}
    }
\end{frame}

\begin{frame}
    \frametitle{Cancelling a Sprint}
    When a Sprint is cancelled:
    \vspace{1em}
    \begin{itemize}
        \setlength\itemsep{0.7em}
        \item Any completed and ``Done'' Product Backlog items are reviewed
        \item If part of the work is potentially releasable, the Product Owner typically accepts it
        \item All incomplete Product Backlog Items are re-estimated and put back on the Product Backlog
    \end{itemize}
\end{frame}

\begin{frame}
    \frametitle{Sprint Review}
    \begin{itemize}
        \setlength\itemsep{0.7em}
        \item During the Sprint Review, the Scrum Team and stakeholders collaborate about what was done in the Sprint
        \item An informal meeting
        \item The presentation of the Increment is intended to elicit feedback and foster collaboration
    \end{itemize}
\end{frame}

\begin{frame}
    \frametitle{Sprint Review}
    \begin{itemize}
        \setlength\itemsep{0.7em}
        \item<1-> \textbf{Step 1}\\
        The Product Owner explains what Product Backlog items have been ``Done'' and what has not been ``Done''
        \item<2-> \textbf{Step 2}\\
        The Development Team discusses what went well during the Sprint, what problems it ran into and how those problems were solved
        \item<3-> \textbf{Step 3}\\
        The Development Team demonstrates the work that it has ``Done'' and answers questions about the Increment
    \end{itemize}
\end{frame}

\begin{frame}
    \frametitle{Sprint Review}
    \begin{itemize}
        \setlength\itemsep{0.7em}
        \item<1-> \textbf{Step 4}\\
        The Product Owner discusses the Product Backlog as it stands. If necessary, they project likely target and delivery dates based on progress to date
        \item<2-> \textbf{Step 5}\\
        The entire group collaborates on what to do next, so that the Sprint Review provides valuable input to subsequent Sprint Planning
    \end{itemize}
\end{frame}